\vspace*{20mm}
{
\Large\bf
\begin{center}
Notation
\end{center}
}

\label{sec:Notation}
Throughout this thesis, the following conventions will be used for
typesetting mathematics unless otherwise indicated:

\begin{itemize}

\item \textbf{Sets} are written in calligraphic type: $\mathcal{S}$.
\item \textbf{Fields} are written in upper case: $\mathbb{F}$. In
particular, $\Reals$ denotes the reals and $\Ints$ denotes the integers.
\item \textbf{Vectors} are written in lower case bold: $\vect{x}$. Vectors are usually column vectors,
with elements specified by subscript index
(\eg $\vect{x}=(x_1,x_2,x_3)$).
\item \textbf{Matrices} are written in upper case: $H$.
The entry in the $i$\th row and $j$\th column is
$H_{i,j}$.
\item \textbf{Conditional probabilities} are written $P(A ~|~ B)$. In
most cases $A$ and $B$ will correspond to the event that certain
variables take on certain values, for example: $P(x ~|~ y)$.
\item \textbf{Expectations} of $x$ with respect to a distribution $p$
are written $\Expected_p[x]$.
\item \textbf{Gradients} of $y$ with respect to $x$ are written $\Deriv{y}{x}$.
\item \textbf{Inner products} for geometric quantities are written
$\vect{x}\cdot\vect{y}$. In the more general context of Hilbert spaces
we switch to the notation $\bigl\langle\vect{x},\vect{y}\bigr\rangle$.
\item Where there is a need to differentiate the estimated and true
values of a quantity $y$, the former will be written $\hat{y}$ and the
latter $y^*$.

\end{itemize}
